\documentclass{article}
\usepackage[T1]{fontenc}
\usepackage{graphicx}
\usepackage{amsmath, amsthm, amssymb}
\usepackage{hyperref}
\usepackage{polski}
\usepackage{array}
\usepackage{float}
\newtheorem{theorem}{Twierdzenie}
\usepackage[a4paper, left=1.5cm, right=3cm, top=2cm, bottom=2cm]{geometry}
\theoremstyle{definition}
\usepackage{enumerate}
\usepackage{listings}
\usepackage{color}
\newtheorem{definition}{Definicja}

\title{Dokumentacja}
\author{Amelia Dorożko, 282259 \\
	Matylda Mordal, 282240 \\ 
	Zuzanna Pawlik, 282230 \\ 
	Paweł Solecki, 282246 \\ 
	Zofia Stępień, 282254}
\date{\today}

\begin{document}
	
	\maketitle
	
	\section{Spis technologii}
	
	\subsection{Użyte biblioteki i pakiety Python}
	\begin{itemize}
		\item pylatex
		\item mysql.connector
		\item numpy
		\item matplotlib
		\item pandas
		\item faker
		\item random
		\item decimal
		\item datetime
		\item subprocess
		\item re
		\item unicodedata
	\end{itemize}
	
	\subsection{Narzędzia i technologie zewnętrzne}
	\begin{itemize}
		\item MySQL
		\item Python
		\item Jupyter Notebook
		\item Visual Studio Code
		\item LaTeX
		\item Conda
	\end{itemize}
	
	\section{Lista plików}
	\begin{itemize}
		\item \textbf{Schemat.json.vuerd} -- Plik zawierający gotowy schemat bazy danych.
		\item \textbf{Raport.py} -- Zawiera analizę danych i wykresy.
		\item \textbf{Dokumentacja.pdf} -- Zawiera opis działania projektu oraz użytych technologii.
		\item \textbf{Uzupelnienie\_tabel.ipynb} -- Plik odpowiedzialny za wstawienie danych testowych do bazy.
	\end{itemize}
	
	\section{Kolejność i sposób uruchamiania plików}
	\subsection{Połączenie z bazą danych}
	\begin{itemize}
		\item Na początku należy połączyć się z bazą.
		\item W oknie konfiguracji połączenia wprowadź odpowiednie dane:
		\begin{itemize}
			\item \textbf{Login}: team19
			\item \textbf{Hasło}: te@mzaiq
			\item \textbf{Baza}: team19
		\end{itemize}
		\item Po wpisaniu wszystkich danych kliknij przycisk "Connect". SQLTools nawiąże połączenie z bazą danych. Jeśli wszystkie dane zostały podane poprawnie, połączenie zostanie zrealizowane.
	\end{itemize}
	
	\subsection{Schemat}
	\textbf{Sposób uruchomienia:}
	\begin{itemize}
		\item Na początku należy zainstalować rozszerzenie ERD Editor.
		\item Następnie pobierz plik schematu bazy danych w formacie \textit{.json.vuerd}.
		\item Za pomocą ERD Editor przekształć na schemat SQL.
		\item Ostatecznie, uruchom wygenerowany kod w pliku \textit{.sql}, będąc połączonym z bazą danych.
	\end{itemize}
	
	\subsection{Uzupełnienie tabel}
	\textbf{Sposób uruchomienia:}
	\begin{itemize}
		\item Uruchom plik w środowisku Jupyter Notebook lub jako skrypt Python.
		\item Skrypt korzysta z biblioteki Faker do generowania przykładowych danych.
		\item Przed uruchomieniem upewnij się, że masz zainstalowane wymagane biblioteki: \texttt{pip install faker mysql-connector-python}.
	\end{itemize}
	
	\subsection{Raport}
	\textbf{Sposób uruchomienia:}
	\begin{itemize}
		\item Otwórz plik w Jupyter Notebook lub uruchom jako skrypt Python.
		\item Wykorzystuje biblioteki pandas, matplotlib, numpy, pylatex – upewnij się, że są zainstalowane: \texttt{pip install pandas matplotlib numpy pylatex}.
	\end{itemize}
	
	\section{Schemat projektu bazy danych}
		\begin{figure}[H]    
		\centering
		\includegraphics[width=0.9\textwidth]{Schemat1.png} 
		\caption{Schemat bazy danych - wersja 1}
	\end{figure}
	
	\begin{figure}[H]    
		\centering
		\includegraphics[width=0.9\textwidth]{Schemat2.png} 
		\caption{Schemat bazy danych - wersja 2}
	\end{figure}
	
	
	\section{Lista zależności funkcyjnych}
	\begin{itemize}
	\item Tabela Klienci
	Klucz główny: id\_klienci
	
	Zależności funkcyjne:
	id\_klienci → {Imie, Nazwisko, Data\_urodzenia, Numer\_telefonu, Email, Dane\_kontaktowe\_bliskich, Ulica, Numer\_lokalu, Miasto, Kod\_pocztowy}
	
	Wszystkie atrybuty są w pełni zależne od klucza głównego.
	
	\item Tabela Pracownicy
	Klucz główny: id\_pracownicy
	
	Zależności funkcyjne:
	id\_pracownicy → {id\_stanowisko, Imie, Nazwisko, Numer\_telefonu, Email, Ulica, Numer\_lokalu, Miasto, Kod\_pocztowy, Pensja, Data\_urodzenia, Data\_zatrudnienia}
	id\_stanowisko → {Kategoria, Koszt\_stanowiska}
	
	Tranzytywna zależność id\_pracownicy → id\_stanowisko → Kategoria jest wydzielona poprzez tabelę Stanowisko, więc struktura jest poprawna.
	
	\item Tabela Rodzaje\_wycieczek
	Klucz główny: id\_rodzaj
	
	Zależności funkcyjne:
	id\_rodzaj → {Nazwa, Opis, id\_wylot, id\_przylot}
	id\_wylot → Miasto 
	id\_przylot → Miasto 
	
	Tranzytywne zależności id\_rodzaj → id\_wylot → Miasto oraz id\_rodzaj → id\_przylot → Miasto są wydzielone poprzez tabele Miejsce\_wylotu i Miejsce\_przylotu.
	
	\item Tabela Realizowane\_wycieczki
	Klucz główny: id\_wycieczki
	
	Zależności funkcyjne:
	id\_wycieczki → {id\_rodzaj, id\_koszty, Cena\_za\_osobe, Data\_rozpoczecia, Data\_zakonczenia}
	id\_koszty → {Nocleg\_za\_noc, Przewodnik, Loty, Atrakcje, Ubezpieczenie, Transport} 
	
	Zależności są poprawnie wydzielone poprzez tabelę Koszty\_organizacji.
	
	\item Tabela Rezerwacje
	Klucz główny: id\_rezerwacje
	
	Zależności funkcyjne:
	id\_rezerwacje → {id\_klienci, id\_pracownicy, id\_wycieczki, id\_transakcje, Data\_rezerwacji}
	Klucze obce (id\_klienci, id\_pracownicy, id\_wycieczki, id\_transakcje) odnoszą się do kluczy głównych w innych tabelach.
	
	Brak zależności funkcyjnych poza kluczami głównymi i obcymi. 
	
	\item Tabela Koszty\_organizacji
	Klucz główny: id\_koszty
	
	Zależności funkcyjne:
	id\_koszty → {Nocleg\_za\_noc, Przewodnik, Loty, Atrakcje, Ubezpieczenie, Transport}.
	
	Wszystkie atrybuty są w pełni zależne od klucza głównego.
	
	\item Tabela Transakcje\_finansowe
	Klucz główny: id\_transakcje
	
	Zależności funkcyjne:
	id\_transakcje → {Zrealizowano, id\_metody}
	id\_metody → Metoda 
	
	Tranzytywna zależność id\_transakcje → id\_metody → Metoda jest obsłużona przez tabelę Metody\_platnosci.
	
	\item Tabela Oceny
	Klucz główny: id\_oceny
	
	Zależności funkcyjne:
	id\_oceny → {id\_rezerwacje, Ocena, Data\_oceny}
	id\_rezerwacje → {id\_klienci, ...}
	
	Zależności są poprawnie wydzielone poprzez tabelę Rezerwacje.
	
	\end{itemize}

	\section{EKNF}
	Uzasadnienie spełnienia postaci EKNF :
	\begin{itemize}
	
	\item  Każda tabela posiada pojedynczy klucz główny (np. `id\_klienci`, `id\_koszty`), który jest elementarny i jednoznacznie identyfikuje rekordy. 
	 
	*Przykład: Klucz `id\_klienci` w tabeli `Klienci` determinuje wszystkie pozostałe atrybuty.
	
	\item  Atrybuty niebędące kluczami zależą wyłącznie od klucza głównego, a nie od innych atrybutów.  
	
	*Przykład: W tabeli `Klienci` atrybut `Email` nie wpływa na `Numer\_telefonu`.
	
	\item  Wszystkie niebanalne zależności funkcyjne są oparte na kluczach głównych.  
	
	*Przykład: W tabeli `Rezerwacje` klucz `id\_rezerwacje` determinuje `id\_klienci`, `id\_pracownicy` itd.
	
	\item  Relacje między tabelami realizowane są przez klucze obce, które nie wprowadzają zależności częściowych ani przechodnich.  
	
	*Przykład: `id\_rodzaj` w tabeli `Realizowane\_wycieczki` odnosi się do klucza głównego w `Rodzaje\_wycieczek`.
	
	\end{itemize}

	\section{Najtrudniejsze elementy}
		
	Największym wyzwaniem okazało się zapewnienie spójności i realizmu danych czasowych w strukturze bazy. Konieczne było precyzyjne dostosowanie wartości dat do kontekstu oraz zachowanie odpowiednich zależności czasowych pomiędzy tabelami. Każdy wpis musiał tworzyć wiarygodną chronologię zdarzeń. Drugim istotnym problemem było opracowanie mechanizmu losowania danych. Konieczne było tworzenie zestawów zróżnicowanych danych przy zachowaniu ich merytorycznej poprawności. 
	
	
\end{document}